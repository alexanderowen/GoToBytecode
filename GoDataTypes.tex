\documentclass{article}

\usepackage[margin=1in]{geometry}
\usepackage{hyperref}
\hypersetup{
    colorlinks=true,
    linkcolor=blue,
    filecolor=magenta,      
    urlcolor=cyan,
}
\usepackage[usenames, dvipsnames]{xcolor}
\definecolor{code}{RGB}{220,220,220}


\begin{document}
\tableofcontents
\newpage

\section{Enumeration of types}
	\subsection{Basic types}
		\begin{itemize}
			\item bool
			\item string
			\item uint
			\item uint8
			\item uint16
			\item uint32
			\item uint64
			\item uintptr
			\item int
			\item int8
			\item int16
			\item int32
			\item int64
			\item float32
			\item float64
			\item complex64
			\item complex128
			\item byte (alias for uint8)
			\item rune (alias for int32)
		\end{itemize}
	\subsection{Composite types}
		\begin{itemize}
			\item array
			\item struct
			\item pointer
			\item function
			\item interface
			\item slice
			\item map
			\item channel
		\end{itemize}

\section{Description of basic types}
	\subsection{Boolean}
		\begin{itemize}
			\item The predeclared boolean type is \colorbox{code}{bool}.
			\item Two predeclared constants: \colorbox{code}{true} and \colorbox{code}{false}.
		\end{itemize}
		\subsubsection{Operators}
			\begin{verbatim}
		    			==    equal
	    				!=    not equal
		
	    		&&    conditional AND    
    			||    conditional OR    
    			!     NOT               
			\end{verbatim}
			
	\subsection{String}
		\begin{itemize}
			\item The \colorbox{code}{string} type is a sequence of bytes, roughly equivalent to a read-only slice of bytes. 
			\item Strings are immutable and have a constant length. 
			\item The length of the string can be accessed with the builtin \colorbox{code}{len} function. Note that this returns the number of bytes in the string, not characters. 
			\item The zero-value for strings is \colorbox{code}{""}.
			\item See \url{https://blog.golang.org/strings}
		\end{itemize} 
		\subsubsection{Operators}
			\begin{verbatim}
	    				+ 	   concatenation
					
		    			==    equal
	    	!=    not equal
	    	<     less
	    	<=    less or equal
	    	>     greater
	    	>=    greater or equal
			\end{verbatim}
			\begin{itemize}
				\item The comparisons '$<$', '$<=$', '$>$', '$>=$' are performed lexicographically, byte-wise. 
			\end{itemize}
	
	\subsection{Unsigned integer types}	
		\subsubsection{Types}
		\begin{itemize}
    			\item uint
				\begin{itemize}
					\item Predeclared numeric type. Either 32 or 64 bits. 
				\end{itemize}
    			\item uint8
   	 		\item uint16
    			\item uint32
    			\item uint64
    			\item uintptr
				\begin{itemize}
					\item An unsigned integer large enough to store the uninterpreted bits of a pointer value. Used to subvert type system. 
					\item See \url{https://golang.org/pkg/unsafe/}
				\end{itemize}
		\end{itemize}
		\subsubsection{Binary operators}
			\begin{verbatim}
				    +    sum  
				    -    difference   
				    *    product       
				    /    quotient     
				    %    remainder  

				    &    bitwise AND    
				    |    bitwise OR    
				    ^    bitwise XOR   
				    &^   bit clear (AND NOT) 

				    <<   left shift
				    >>   right shift
				    
	    ==    equal
	    !=    not equal
	    <     less
	    <=    less or equal
	    >     greater
	    >=    greater or equal
			\end{verbatim}	
			\begin{itemize}
				\item The operands MUST be the same type i.e. cannot add uint and uint32, even if they are the same number of bits.  
			\end{itemize}
		\subsubsection{Unary operators}
			\begin{verbatim}
		   		+x                     
	    -x    negation         
	    ^x    bitwise complement 
			\end{verbatim}
		\subsubsection{Overflow}
			\begin{itemize}
				\item Constant declarations will raise an exception on overflow i.e. \begin{verbatim}var x uint8 = 300\end{verbatim} will raise an exception.
				\item The operations +, -, *, and \(<<\) may overflow unsigned integers. The operations are computed modulo \(2^X\), with 'X' as in 'uintX'. As a result, integer overflow wraparound may occur. 
			\end{itemize}
		\subsubsection{Notes}
			\begin{itemize}
				\item The size (in bits) of integers is 'X' in 'uintX'. 'uint8' is 8 bits, 'uint16' is 16 bits, and so on. 
				\item The range of values for 'uintX' is [0, \(2^X - 1\)].
				\item The size of 'uint' is either 32 or 64 bits. The size is implementation specific. It can be expected to be the word size of the CPU, but this is NOT guaranteed.
				\item The zero-value for unsigned integers is \colorbox{code}{0}.
			\end{itemize} 
			
			
    	\subsection{Signed integer types}
			\subsubsection{Types}
			\begin{itemize}
    				\item int
					\begin{itemize}
						\item Predeclared numeric type. Either 32 or 64 bits. 
					\end{itemize}
    				\item int8
    				\item int16
   	 			\item int32
  	  			\item int64
			\end{itemize}
			\subsubsection{Binary operators}
				\begin{verbatim}
					    +    sum  
					    -    difference   
					    *    product       
					    /    quotient     
					    %    remainder  

					    &    bitwise AND    
					    |    bitwise OR    
					    ^    bitwise XOR   
					    &^   bit clear (AND NOT) 

					    <<   left shift (NOTE BELOW)
					    >>   right shift (NOTE BELOW)
					    
	    ==    equal
	    !=    not equal
	    <     less
	    <=    less or equal
	    >     greater
	    >=    greater or equal
				\end{verbatim}	
				\begin{itemize}
					\item The operands MUST be the same type i.e. cannot add int and int32, even if they are the same number of bits.  
					\item NOTE: The right operand of left and right shift must be an unsigned integer
					\item As expected, right shift retains the signed'ness. Right shift by 1 is equivalent to division by 2.
				\end{itemize}
			\subsubsection{Unary operators}
				\begin{verbatim}
		   		+x                     
	    -x    negation         
	    ^x    bitwise complement 
				\end{verbatim}
			\subsubsection{Overflow}
				\begin{itemize}
					\item Constant declarations will raise an exception on overflow i.e. \begin{verbatim}var x int8 = 300\end{verbatim} will raise an exception.
					\item The operations +, -, *, and \(<<\) may legally overflow signed integers. 
				\end{itemize}
			\subsubsection{Notes}
				\begin{itemize}
					\item Signed integers are represented using two's-complement. 
					\item The size of 'int' is either 32 or 64 bits. The size is implementation specific. It can be expected to be the word size of the CPU, but this is NOT guaranteed.
					\item The zero-value for signed integers is \colorbox{code}{0}.
				\end{itemize}
				
				
    	\subsection{Floating-point types}
			\subsubsection{Types}
			\begin{itemize}
    				\item float32
    				\item float64
			\end{itemize}
			\subsubsection{Operators}
				\begin{verbatim}
					    +    sum  
					    -    difference   
					    *    product       
					    /    quotient     
					    
	    ==    equal
	    !=    not equal
	    <     less
	    <=    less or equal
	    >     greater
	    >=    greater or equal
				\end{verbatim}	
				\begin{itemize}
					\item The operands MUST be the same type i.e. cannot add float32 and float64.
					\item Floating-point comparison operators '$==$', '$!=$', '$<$', '$<=$', '$>$', and '$>=$' are performed as defined by the IEEE-754 standard. 
				\end{itemize}
			\subsubsection{Notes}
				\begin{itemize}
					\item Follows the IEEE-754 standard for floating points. 
					\item The size of 'floatX' is 'X' bits. 
					\item The zero-value for floating points is \colorbox{code}{0.0}.
				\end{itemize}
				
				
    		\subsection{Complex-number types}
			Builtin support for complex numbers.
			\subsubsection{Types}
			\begin{itemize}
    				\item complex32
					\begin{itemize}
						\item Uses float32 for real and imaginary parts.
					\end{itemize}
    				\item complex64
					\begin{itemize}
						\item Uses float64 for real and imaginary parts.
					\end{itemize}
			\end{itemize}
			\subsubsection{Operators}
				\begin{verbatim}
					    +    sum  
					    -    difference   
					    *    product       
					    /    quotient     
					    
	    ==    equal
	    !=    not equal
				\end{verbatim}	
				\begin{itemize}
					\item The operands MUST be the same type i.e. cannot add complex32 and complex64.
					\item Two complex variables x and y are equal if real(x) == real(y) and imag(x) == imag(y).
				\end{itemize}
			\subsubsection{Notes}
				\begin{itemize}
					\item Typically constructed using the builtin \colorbox{code}{complex} function. See \url{https://golang.org/pkg/builtin/#complex}
					\item Real and imaginary parts can be accessed using the builtin \colorbox{code}{real} and \colorbox{code}{imag} functions. See \url{https://golang.org/pkg/builtin/#real} and \url{https://golang.org/pkg/builtin/#imag}
					\item Additional function support provided in the "math/cmplx" package. See \url{https://golang.org/pkg/math/cmplx/}
				\end{itemize}
			
			
    		\subsection{Aliased types}
			\begin{itemize}
    				\item byte (alias for uint8)		
    				\item rune (alias for int32)
				\begin{itemize}
					\item An integer value identifying a Unicode code point.
				\end{itemize}
			\end{itemize}
	

\section{Description of composite types}
	\subsection{Array}
		A sequence of elements of a certain type. For example, \colorbox{code}{[32]int8} is an array of int8 with 32 elements. 
		\begin{itemize}
			\item Arrays are copy on assignment. When assigned to a new variable, all elements are copied over.
			\item Arrays are pass by value. When passed into a function, a new array is created and passed in. 
			\item Upon creation, the elements of an array are set to their zero-value. For example, \colorbox{code}{[2]float32} will create an array \colorbox{code}{[0.0, 0.0]}.
		\end{itemize}
		\subsubsection{Operators}
		\begin{verbatim}
	    ==    equal
	    !=    not equal
		\end{verbatim}
		\begin{itemize}
			\item Two arrays can be compared with '==' and '!=' if their element types are comparable with '==' and '!='. 
			\item Two array values are equal if their corresponding elements are equal.
		\end{itemize}
		
	\subsection{Struct}
		A collection of named and unnamed fields, each of which has a type. Similar to a C struct.
		\begin{itemize}
			\item Upon creation, the fields of a struct are set to their zero-value. 
			\item See \url{https://golang.org/ref/spec#Struct_types}
		\end{itemize}
		\subsubsection{Operators}
			\begin{verbatim}
	    ==    equal
	    !=    not equal
			\end{verbatim}
			\begin{itemize}
				\item "Struct values are comparable if all their fields are comparable. Two struct values are equal if their corresponding non-blank fields are equal."
				\item Does not support operator overloading. 
			\end{itemize}
		
	\subsection{Pointer}
		A pointer to an address in memory for a certain type. 
		\begin{itemize}
			\item The zero-value for pointers is \colorbox{code}{nil}. 
		\end{itemize}
		\subsubsection{Operators}
			\begin{verbatim}
	    *     dereference
			
	    ==    equal
	    !=    not equal
			\end{verbatim}
			\begin{itemize}
				\item Two pointers are equivalent if they point to the same variable or they are both \colorbox{code}{nil}. 
				\item Does not support pointer arithmetic.
			\end{itemize}
		
	\subsection{Function}	
		\begin{itemize}
			\item Functions are first-class citizens. They can be assigned to variables, returned from other functions, and so on.
			\item The zero-value for functions is \colorbox{code}{nil}.
			\item Cannot be compared to other functions. Exception: functions can be compared to \colorbox{code}{nil} with '==' and '!='.
			\item For examples, see \url{https://golang.org/doc/codewalk/functions/}
			\item For reference specification, see \url{https://golang.org/ref/spec#Function_types}
		\end{itemize}
		
	\subsection{Interface}
		A collection of function signatures that defines an interface. 
		\begin{itemize}
			\item A type is said to implement an interface if it has a method for each function signature.
			\item The empty interface, \colorbox{code}{interface\{\}} is implemented by all types. 
			\item The zero-value for interfaces is \colorbox{code}{nil}.
			\item See \url{https://golang.org/ref/spec#Interface_types}
		\end{itemize}
		\subsubsection{Operators}
			\begin{verbatim}
				    ==    equal
				    !=    not equal
			\end{verbatim}
			\begin{itemize}
				\item "Two interface values are equal if they have identical dynamic types and equal dynamic values or if both have value nil."
			\end{itemize}
		
	\subsection{Slice}
		A slice consists of a reference to the underlying array, the length of the slice, and the capacity of the underlying array. 				
		\begin{itemize}
			\item Use the builtin functions \colorbox{code}{cap} and \colorbox{code}{len} to get the capacity and length.
			\item Cannot be compared to other slices. Exception: slices can be compared to \colorbox{code}{nil} with '==' and '!='.
			\item See \url{https://blog.golang.org/go-slices-usage-and-internals}
		\end{itemize}
				
	\subsection{Map}
		Unordered collection of key:value pairs. 
		\begin{itemize}
			\item The key type must have \colorbox{code}{==} and \colorbox{code}{!=} defined. 
			\item The zero-value for maps is \colorbox{code}{nil}.
			\item Typically initialized with the builtin \colorbox{code}{make} function. For example, \colorbox{code}{var m map[int]string = make(map[int]string)}
			\item Cannot be compared to other maps. Exception: maps can be compared to \colorbox{code}{nil} with '==' and '!='.
			\item See \url{https://blog.golang.org/go-maps-in-action}
		\end{itemize}
		
	\subsection{Channel}
		Mechanism for message-passing. To be used by concurrent applications.
		\begin{itemize}
			\item A channel consists of a direction (send/receive/both), the type it handles, and the capacity for its buffer. 
			\item The zero-value for channels is \colorbox{code}{nil}.
			\item See \url{https://golang.org/ref/spec#Channel_types}
		\end{itemize}
		\subsubsection{Operators}
			\begin{verbatim}
				    <-    Receive operator
				    
				    ==    equal
				    !=    not equal
			\end{verbatim}
			\begin{itemize}
				\item For information about the receive operator, see \url{https://golang.org/ref/spec#Receive_operator}
				\item Two channels are equal if they were created by the same call to \colorbox{code}{make} or they are both \colorbox{code}{nil}.
			\end{itemize}
	
\section{Other Notes}
	\subsection{Zero values}
		Variables cannot be uninitialized. Whenever a variable is created, it is initialized to the type's zero value. The zero values of the builtin types are listed below. 
		\begin{center}
			\begin{tabular}{ | c | c | }
			\hline
 			Type & Zero Value  \\ 
			\hline
 			All integer types & 0  \\  
 			Floating points & 0.0   \\
			Booleans & false \\
			Strings & "" \\
			Pointers & nil \\
			Functions & nil \\
			Interfaces & nil \\
			Slices & nil \\
			Channels & nil \\
			Maps & nil \\
			\hline
			\end{tabular}
		\end{center}
		Initialization is recursive, so a struct is initialized with all of it's fields set to their zero values. The same goes for complex types, where the real and imaginary values are set to zero.
		See \url{https://golang.org/ref/spec#The_zero_value}	
		
	\subsection{Nil}
	\colorbox{code}{nil} is the zero-value for reference types, which includes pointers, functions, interfaces, slices, channels, and maps. Nil has no type. 
		
	\subsection{Error}
	\colorbox{code}{error} is the builtin interface type that specifies any error condition. See \url{https://golang.org/pkg/builtin/#error}.
		
	\subsection{Precedence}
	\begin{verbatim}
Precedence    Operator
    5             *  /  %  <<  >>  &  &^
    4             +  -  |  ^
    3             ==  !=  <  <=  >  >=
    2             &&
    1             ||
	\end{verbatim}
	See \url{https://golang.org/ref/spec#Operators}
	\subsection{Memory allocation}
		When possible, the Go compilers will allocate variables that are local to a function in that function's stack frame. However, if the compiler cannot prove that the variable is not referenced after the function returns, then the compiler must allocate the variable on the garbage-collected heap to avoid dangling pointer errors. See \url{https://golang.org/doc/faq#stack_or_heap}
		
	\subsection{Type conversions}
		See \url{https://golang.org/ref/spec#Conversions}

\end{document}