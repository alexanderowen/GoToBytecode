\documentclass[12pt]{article}


\usepackage{fancyvrb}
%\usepackage[margin=1in]{geometry}
\usepackage{hyperref}
\hypersetup{
    colorlinks=true,
    linkcolor=blue,
    filecolor=magenta,      
    urlcolor=cyan,
}

\usepackage[usenames, dvipsnames]{xcolor}
\definecolor{code}{RGB}{220,220,220}

\newcommand*{\formalfont}{\fontfamily{ptm}\selectfont}

\begin{document}
	\tableofcontents
	\newpage
	
	\section{.gobc File Format}
		This document describes the Golang bytecode file format, {\formalfont .gobc}. Each {\formalfont .gobc} file contains the bytecode for a Go source file. \\ \\
		A class file consists of a stream of 8-bit bytes. All 16-bit, 32-bit, and 64-bit quantities are constructed by reading in two, four, and eight consecutive 8-bit bytes, respectively. \\ \\
		This document uses a short-hand for specifying the number of bytes associated with the data. The types u8, u16, u32, and u64 represent a one-, two-, four- or eight-byte quantity, respectively. \\ \\
		This document presents the {\formalfont .gobc} file format using pseudo-code of C syntax. Arrays are zero-indexed. 
		
		\subsection{The gobcFile structure}
			A {\formalfont .gobc} file consists of a single {\formalfont gobcFile} structure: \\ 
			\begin{Verbatim}[frame=single]
gobcFile {	
	u32           magicNumber
	u32           functionCount
	functionInfo  functions[functionCount]
}
			\end{Verbatim}
			The items that appear in the {\formalfont gobcFile} structure are defined below: \\ \\
			\textbf{magicNumber} \\
			The {\formalfont magicNumber} item supplies the magic number identifying the {\formalfont gobc} file format; it has the value \colorbox{code}{0xCAFEDEAD}. \\ \\
			\textbf{functionCount} \\
			The {\formalfont functionCount} item specifies the number of functions defined in the file in the global scope. \\ \\
			\textbf{functions} \\
			The {\formalfont functions} item is an array where each element is a {\formalfont functionInfo} structure giving a complete description of the function. 
		
	

\end{document}