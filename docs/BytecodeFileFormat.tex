\documentclass[12pt]{article}


\usepackage{fancyvrb}
%\usepackage[margin=1in]{geometry}
\usepackage{hyperref}
\hypersetup{
    colorlinks=true,
    linkcolor=blue,
    filecolor=magenta,      
    urlcolor=cyan,
}

\usepackage[usenames, dvipsnames]{xcolor}
\definecolor{code}{RGB}{220,220,220}

\newcommand*{\formalfont}{\fontfamily{ptm}\selectfont}
\newcommand\tab[1][1cm]{\hspace*{#1}}

\begin{document}
	\textbf{Disclaimer} \\ This document is not comprehensive of all Golang features. 
	\tableofcontents
	\newpage
	
	\section{.gobc File Format}
		This document describes the Golang bytecode file format, {\formalfont .gobc}. Each {\formalfont .gobc} file contains the bytecode for a Go source file. \\ \\
		A class file consists of a stream of 8-bit bytes. All 16-bit, 32-bit, and 64-bit quantities are constructed by reading in two, four, and eight consecutive 8-bit bytes, respectively. \\ \\
		This document uses a short-hand for specifying the number of bytes associated with the data. The types u8, u16, u32, and u64 represent a one-, two-, four- or eight-byte quantity, respectively. \\ \\
		This document presents the {\formalfont .gobc} file format using pseudo-code of C syntax. Arrays are zero-indexed. 
		
		\subsection{gobcFile structure}
			A {\formalfont .gobc} file consists of a single {\formalfont gobcFile} structure: \\ 
			\begin{Verbatim}[frame=single]
gobcFile {	
	u32           magicNumber
	u64           functionCount
	functionInfo  functions[functionCount]
}
			\end{Verbatim}
			The items that appear in the {\formalfont gobcFile} structure are defined below: \\ \\
			\textbf{magicNumber} \\
				\tab The {\formalfont magicNumber} item supplies the magic number identifying the {\formalfont gobc} file format; it has the value \colorbox{code}{0xCAFEDEAD}. \\ \\
			\textbf{functionCount} \\
				\tab The {\formalfont functionCount} item specifies the number of functions defined in the file in the global scope. \\ \\
			\textbf{functions} \\
				\tab The {\formalfont functions} item is an array where each element is a {\formalfont functionInfo} structure giving a complete description of the function. 
			
		\subsection{functionInfo structure}
			Each Golang function is described by a {\formalfont functionInfo} structure. \\ \\
			\textbf{TODO}: How to handle anonymous and first-class functions? \\ \\
			\begin{Verbatim}[frame=single]
functionInfo {	
	u8              accessFlags
	u64             nameLength
	u8	      name[nameLength]
	u32	     attributesCount
	attributesInfo  attributes[attributesCount]
}
			\end{Verbatim}
			The items that appear in the {\formalfont functionInfo} structure are defined below: \\ \\
			\textbf{accessFlags} \\
			\tab The value of the accessFlags item is a mask of flags used to denote access permission to and properties of this function. The interpretation of each flag, when set, is shown below.
			\begin{center}
			\begin{tabular}{ |c|c|c| } 
				\hline 
				Flag & Value & Description \\ [0.5ex] 
 				\hline \hline
				EXPORTED & 0x0001 & Exported function; can be accessed outside the package \\
				\hline 
			\end{tabular}
			\end{center} 
			\textbf{nameLength} \\
				\tab The {\formalfont nameLength} item specifies the number of bytes that comprise the function name. \\ \\
			\textbf{name} \\
				\tab The {\formalfont name} item is an array of bytes that comprise the function name. \\ \\
			\textbf{attributesCount} \\
				\tab The {\formalfont attributesCount} item specifies the number of attributes of this function. \\ \\
			\textbf{attributes} \\
				\tab The {\formalfont attributes} item is an array where each element is an {\formalfont attributesInfo} structure specifying an attribute. \\ \\
				The {\formalfont functionInfo} structure can contain the following attribute structures:
				\begin{itemize}
					\item {\formalfont codeAttribute}
				\end{itemize}
				
		\subsection{attributeInfo structures}
			Attributes are attached to various structures to provide more detailed information. All attributes share the following general structure: \\ \\
			\begin{Verbatim}[frame=single]
attributeInfo {	
	u8        attributeType
	u64       attributeLength
	u8        data[attributesLength]
}
			\end{Verbatim}
			The items that appear in the {\formalfont attributeInfo} structure are defined below: \\ \\
			\textbf{attributeType} \\
				\tab The {\formalfont attributeType} item specifies which attribute the structure represents. \\ \\
			\textbf{attributeLength} \\
				\tab The {\formalfont attributeLength} item specifies the length of the subsequent information in bytes. The length does not include the initial nine bytes that contain the attributeType and attributeLength items. \\ \\
			\textbf{data} \\
				\tab The {\formalfont data} item contains the data of the attribute. Each attribute will have a different structure for this data. \\ \\
			
			\subsubsection{codeAttribute structure}
				The code attribute is a variable-length attribute in the attributes table of a functionInfo structure. A code attribute contains the instructions and auxiliary information for a single function.
			\begin{Verbatim}[frame=single]
codeAttribute {	
	u8        attributeType
	u64       attributeLength
	u64       codeLength
	u8        code[codeLength]
}
			\end{Verbatim}
			The items that appear in the {\formalfont codeAttribute} structure are defined below: \\ \\
			\textbf{attributeType} \\
				\tab The {\formalfont attributeType} item specifies which attribute the structure represents. The value for {\formalfont codeAttribute} is \colorbox{code}{0x01} \\ \\
			\textbf{attributeLength} \\
				\tab The {\formalfont attributeLength} item specifies the length of the subsequent information in bytes. The length does not include the initial nine bytes that contain the attributeType and attributeLength items. \\ \\
			\textbf{codeLength} \\
				\tab The {\formalfont codeLength} item specifies the length of the {\formalfont code} item. \\ \\
			\textbf{code} \\
				\tab The {\formalfont code} item is an array which contains the instructions of the function. Each instruction must be read 2 bytes at a time. Each instruction has a deterministic length that can be read, which is given in the instruction specification. 
						
	

\end{document}