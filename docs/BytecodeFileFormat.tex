\documentclass[12pt]{article}


\usepackage{fancyvrb}
%\usepackage[margin=1in]{geometry}
\usepackage{hyperref}
\hypersetup{
    colorlinks=true,
    linkcolor=blue,
    filecolor=magenta,      
    urlcolor=cyan,
}

\usepackage[usenames, dvipsnames]{xcolor}
\definecolor{code}{RGB}{220,220,220}

\newcommand*{\formalfont}{\fontfamily{ptm}\selectfont}
\newcommand\tab[1][1cm]{\hspace*{#1}}

\begin{document}
	\textbf{Disclaimer} \\ This document is not comprehensive of all Golang features. Document based on Chapter 4 of the JVM specification. \href{https://docs.oracle.com/javase/specs/jvms/se8/html/jvms-4.html}{Link}. 
	\tableofcontents
	\newpage
	
	\section{.gobc File Format}
		This document describes the Golang bytecode file format, {\formalfont .gobc}. Each {\formalfont .gobc} file contains the bytecode for a Go source file. \\ \\
		A class file consists of a stream of 8-bit bytes. All 16-bit, 32-bit, and 64-bit quantities are constructed by reading in two, four, and eight consecutive 8-bit bytes, respectively. \\ \\
		This document uses a short-hand for specifying the number of bytes associated with the data. The types u8, u16, u32, and u64 represent a one-, two-, four- or eight-byte quantity, respectively. \\ \\
		This document presents the {\formalfont .gobc} file format using pseudo-code of C syntax. Arrays are zero-indexed. 
		
		\subsection{gobcFile structure}
			A {\formalfont .gobc} file consists of a single {\formalfont gobcFile} structure: \\ 
			\begin{Verbatim}[frame=single]
gobcFile {	
	u32           magicNumber
	u64           functionCount
	functionInfo  functions[functionCount]
}
			\end{Verbatim}
			The items that appear in the {\formalfont gobcFile} structure are defined below: \\ \\
			\textbf{magicNumber} \\
				\tab The {\formalfont magicNumber} item supplies the magic number identifying the {\formalfont gobc} file format; it has the value \colorbox{code}{0xCAFEDEAD}. \\ \\
			\textbf{functionCount} \\
				\tab The {\formalfont functionCount} item specifies the number of functions defined in the file in the global scope. \\ \\
			\textbf{functions} \\
				\tab The {\formalfont functions} item is an array where each element is a {\formalfont functionInfo} structure giving a complete description of the function. 
			
		\subsection{functionInfo structure}
			Each Golang function is described by a {\formalfont functionInfo} structure. \\ \\
			\textbf{TODO}: How to handle anonymous and first-class functions? \\ \\
			\begin{Verbatim}[frame=single]
functionInfo {	
	u8              accessFlags
	u64             nameLength
	u8	      name[nameLength]
	u64             descriptorLength
	u8              descriptor[descriptorLength]
	u32	     attributesCount
	attributesInfo  attributes[attributesCount]
}
			\end{Verbatim}
			The items that appear in the {\formalfont functionInfo} structure are defined below: \\ \\
			\textbf{accessFlags} \\
			\tab The value of the accessFlags item is a mask of flags used to denote access permission to and properties of this function. The interpretation of each flag, when set, is shown below.
			\begin{center}
			\begin{tabular}{ |c|c|c| } 
				\hline 
				Flag & Value & Description \\ [0.5ex] 
 				\hline \hline
				EXPORTED & 0x01 & Exported function; can be accessed outside the package \\
				\hline 
			\end{tabular}
			\end{center} 
			\textbf{nameLength} \\
				\tab The {\formalfont nameLength} item specifies the number of bytes in {\formalfont name}. \\ \\
			\textbf{name} \\
				\tab The {\formalfont name} item is an array of bytes that comprise the function name. It can be evaluated as a string. \\ \\
			\textbf{descriptorLength} \\
				\tab The {\formalfont descriptorLength} item specifies the number of bytes in {\formalfont descriptor}. \\ \\
			\textbf{descriptor} \\
				\tab The {\formalfont descriptor} item is an array of bytes that comprise the function descriptor. The descriptor describes the type of the parameters and the return types. It should be evaluated as a string. See \hyperref[sec:functionDescriptor]{Function Descriptor} for more details.  \\ \\
			\textbf{attributesCount} \\
				\tab The {\formalfont attributesCount} item specifies the number of attributes of this function. \\ \\
			\textbf{attributes} \\
				\tab The {\formalfont attributes} item is an array where each element is an {\formalfont attributesInfo} structure specifying an attribute. \\ \\
				The {\formalfont functionInfo} structure can contain the following attribute structures:
				\begin{itemize}
					\item {\formalfont \hyperref[sec:codeAttribute]{codeAttribute}}
				\end{itemize}
				
			\subsubsection{Function Descriptor}
			\label{sec:functionDescriptor}
				The function descriptor represents the parameters that the function takes and the values that it returns. It should be evaluated as a string. It is described by the following grammar:
				\begin{Verbatim}[frame=single]
functionDescriptor -> "(" parameterDesc* ")" returnDesc* 
parameterDesc -> Type
returnDesc -> Type
Type -> I64     // int64
      | F64     // float64
      | S       // string
      | V       // void
				\end{Verbatim}
				For example, the descriptor \colorbox{code}{(I64S)SI64} takes two arguments, an int64 and a string, and returns two values, a string and an int64.
				
				
		\subsection{attributeInfo structures}
			Attributes are attached to various structures to provide more detailed information. All attributes share the following general structure: \\ \\
			\begin{Verbatim}[frame=single]
attributeInfo {	
	u8        attributeType
	u64       attributeLength
	u8        data[attributesLength]
}
			\end{Verbatim}
			The items that appear in the {\formalfont attributeInfo} structure are defined below: \\ \\
			\textbf{attributeType} \\
				\tab The {\formalfont attributeType} item specifies which attribute the structure represents. \\ \\
			\textbf{attributeLength} \\
				\tab The {\formalfont attributeLength} item specifies the length of the subsequent information in bytes. The length does not include the initial nine bytes that contain the attributeType and attributeLength items. \\ \\
			\textbf{data} \\
				\tab The {\formalfont data} item contains the data of the attribute. Each attribute will have a different structure for this data. \\ \\
			
			\subsubsection{codeAttribute structure}
			\label{sec:codeAttribute}
				The code attribute is a variable-length attribute in the attributes table of a functionInfo structure. A code attribute contains the instructions and auxiliary information for a single function.
			\begin{Verbatim}[frame=single]
codeAttribute {	
	u8        attributeType
	u64       attributeLength
	u64       codeLength
	u8        code[codeLength]
}
			\end{Verbatim}
			The items that appear in the {\formalfont codeAttribute} structure are defined below: \\ \\
			\textbf{attributeType} \\
				\tab The {\formalfont attributeType} item specifies which attribute the structure represents. The value for {\formalfont codeAttribute} is \colorbox{code}{0x01} \\ \\
			\textbf{attributeLength} \\
				\tab The {\formalfont attributeLength} item specifies the length of the subsequent information in bytes. The length does not include the initial nine bytes that contain the attributeType and attributeLength items. \\ \\
			\textbf{codeLength} \\
				\tab The {\formalfont codeLength} item specifies the length of the {\formalfont code} item. \\ \\
			\textbf{code} \\
				\tab The {\formalfont code} item is an array which contains the instructions of the function. Each instruction must be read 2 bytes at a time. Each instruction has a deterministic length that can be read, which is given in the instruction specification. 
						
	\newpage
	
	
	\section{The Interpreter}
		A stack-based VM to execute the {\formalfont gobc} instructions. 
		\subsection{Operand Stack}
			When a context for execution is created (i.e. when a function is called), a LIFO stack is created for it, called the operand stack. The operand stack is used to store intermediate values to operate on with bytecode instructions, parameters passed to functions, and return values of functions. \\ \\
			Bytecode instructions wiill push and/or pop onto the operand stack, depending on the instruction. 
		\subsection{Local Variable Table}
			When a function is called, a local variable table needs to be instantiated to hold the local variables. This will be an array that is zero-indexed. When a function returns from execution, the local variable table should be deallocated. \\ \\
			Instructions that use the local variable table include instructions that contain 'store' and 'load' in the name, such as \colorbox{code}{i64store} and \colorbox{code}{i64load}.
		

\end{document}